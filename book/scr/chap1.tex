\part{动力系统的概念}
\section{动力系统}

\begin{Defination}[离散动力系统]
\(f\)是拓扑空间\(X\)上的同胚,
其迭代构成来了一个\textbf{离散动力系统}\(\{f^n:n\in Z\}\).
\end{Defination}

\begin{Defination}[连续动力系统]
\(\phi(t,x):R\times X\mapsto X\)是拓扑空间X上的连续流,
则称之为\textbf{连续动力系统}。
\end{Defination}

\begin{Defination}[微分动力系统]
如果X是一个\(C^r\)微分流形,
\(\phi\)是\(C^r\)映射,
则称\(\phi\)诱导出的动力系统是\textbf{微分动力系统}。
\end{Defination}
\subsection{离散动力系统}

\begin{Defination}[拓扑空间]
设\(X\)是一个集合,\(\mathscr{T}\) 是\(X\)的一个子集族,
如果\(\mathscr{T}\)满足如下条件:
                \begin{enumerate}
                        \item [\((\romannumeral 1)\)]  \( X,\emptyset \in \mathscr{T}\);
                        \item [\((\romannumeral 2)\)]  若\(A,B \in \mathscr{T}\),则\(A\cap B\in \mathscr{T}\);
                        \item [\((\romannumeral 3)\)]  若\(\mathscr{T}_1\subset \mathscr{T}\),则 \(\cup_{A \in \mathscr{T}_1}A\in \mathscr{T}\),
                \end{enumerate}
则称\(\mathscr{T}\)是X的一个拓扑。
\end{Defination}
如果\(\mathscr{T}\)是\(X\)的一个拓扑,
则称偶对\((X,\mathscr{T})\)是一个\textbf{拓扑空间},
或称集合\(X\)是一个相对于拓扑\(\mathscr{T}\)而言的拓扑空间;
或者当拓扑空间\(\mathscr{T}\)已经约定或者在行文中已经指出而无需说明时,
称集合\(X\)是一个拓扑空间。
此外\(\mathscr{T}\)的每一个元素都叫做拓扑空间\((X.\mathscr{T})\)(或\(X\))中的一个开集。

按照布尔巴基学派的观点,在集合上可以定义的母结构主要有三种:


一种是代数结构,集合上有了代数结构之后,就可以运算,从两个元素产生第三种。

% \[\left\{
%       \begin{array}
%       定义加法(适当的运算法则)&\text{群},\\
%       定义加法(适当的运算法则)&\text{环},\\
%       定义加法(适当的运算法则)&\text{域},\\
%       定义加法(适当的运算法则)&\text{线性空间}.
%       \end{array}
% \right.\]

一种是序结构,集合之间的元素有了先后关系。
一种是拓扑结构,它用来描述连续性,分离性,附近,边界等性质。

\begin{Defination}
设X和Y是两个拓扑空间,
\(f:X\mapsto Y\),
如果Y中每一个开集U的原像\(f^{-1}(U)\)是X中的一个开集,
则称f是从X到Y的一个\textbf{连续映射},
或简称映射f连续。
\end{Defination}

\begin{Defination}
设X和Y是两个拓扑空间,
如果\(f:X\mapsto Y\)是一个一一映射,
并且\(f,f^{-1}:X\mapsto Y\),
则称f是一个\textbf{同胚映射}或\textbf{同胚}。
\end{Defination}

\begin{Defination}[离散动力系统]
\(f\)是拓扑空间\(X\)上的同胚,
其迭代构成来了一个离散动力系统\(\{f^n:n\in Z\}\).
\end{Defination}


在\(R^2\)上的线性变换\(R_{\theta}\)是指一个旋转\(\theta\)角的变换,
它显然导出一个动力系统。


\begin{proof}
(1)\(R^2\)是拓扑空间。
我们引入\(R^2\)中的标准拓扑,
由此,我们说明了\(R^2\)上面有拓扑结构,\(R^2\)是拓扑空间。
(2)\(R_{\theta}\)是同胚映射。
在都是空间中,
\(f\)同胚,
当且仅当\(f\)和\(f^{-1}\)都是连续函数。

当绕\((0,0)\)旋转\(\theta\)角时,
我们知道变换矩阵为
\[M(\theta)=
\left(
  \begin{array}{cc}
    \cos \theta & -\sin \theta \\
    \sin \theta & \cos \theta \\
  \end{array}
\right)\]
也就是,变换矩阵是一个连续函数矩阵。
其逆变换为:
\[M(-\theta)=
\left(
  \begin{array}{cc}
    \cos (- \theta) & -\sin (-\theta) \\
    \sin (-\theta) & \cos (-\theta) \\
  \end{array}
\right)\]
也是一个连续函数矩阵。

如果是绕\((x',y')\)旋转,
首先我们将维度升高一维,
将初始点写为\((x_0,y_0,1)\),
变换后的点写为\((x_1,y_1,1)\),
这是如果是绕原点的变换,
那么变换矩阵可以相应的写为:

\[M(\theta)=
\left(
  \begin{array}{ccc}
    \cos \theta & -\sin \theta & 0 \\
    \sin \theta & \cos \theta  & 0 \\
    0 & 0 & 1
  \end{array}
\right)\]

首先我们用一个变换矩阵将其平移到\((0,0)\),
这是变换矩阵和逆变换的变换矩阵为
\[\left(
  \begin{array}{ccc}
    1 & 0 & -x' \\
    0 & 1 & -y' \\
    1 & 0 & 1 \\
  \end{array}
\right),
\left(
  \begin{array}{ccc}
    1 & 0 & x' \\
    0 & 1 & y' \\
    1 & 0 & 1 \\
  \end{array}
\right)\]

这时,
我们将绕绕\((x',y')\)旋转分解成三个变换,
首先平移\((-x',-y')\),
在绕\((0,0)\)旋转,
然后再平移\((x',y')\)将这三个矩阵乘起来就是我们的变换矩阵。
\begin{align*}
  M(\theta,(x',y'))= & \left(
  \begin{array}{ccc}
    1 & 0 & -x' \\
    0 & 1 & -y' \\
    1 & 0 & 1 \\
  \end{array}
\right)
\left(
  \begin{array}{ccc}
    \cos \theta & -\sin \theta & 0 \\
    \sin \theta & \cos \theta  & 0 \\
    0 & 0 & 1\\
  \end{array}
\right)
\left(
  \begin{array}{ccc}
    1 & 0 & x' \\
    0 & 1 & y' \\
    1 & 0 & 1 \\
  \end{array}
\right)\\
  = & \left(
  \begin{array}{ccc}
    \cos \theta & -\sin \theta & (1-\cos\theta)x'+\sin\theta y' \\
    \sin \theta & \cos \theta  & (1-\cos\theta)y'+\sin\theta y' \\
    0 & 0 & 1\\
  \end{array}
\right)
\end{align*}

它是一个连续函数举证,
把\(\theta\)换成\(-\theta\)就得到了逆变换的变换矩阵。
\[\left(
  \begin{array}{ccc}
    \cos (-\theta) & -\sin (-\theta) & (1-\cos(-\theta))x'+\sin(-\theta) y' \\
    \sin (-\theta) & \cos (-\theta)  & (1-\cos(-\theta))y'+\sin(-\theta) y' \\
    0 & 0 & 1\\
  \end{array}
\right)
\]

\end{proof}


\subsection{连续动力系统}

\begin{Defination}
一个映射\(\phi(t,x):R\times X \rightarrow X\)称为集合X上的一个\textbf{流},
如果对\(\forall t_1,t_2 \in R,x\in X\).
\begin{enumerate}
                        \item [\((\romannumeral 1)\)]  \(\phi(0,x)=x\);
                        \item [\((\romannumeral 2)\)]  \(\phi(t_1+t_2,x)=\phi(t_1,\phi(t_2,x))\);
                \end{enumerate}
\end{Defination}


设\(x=\phi(t,x_0)\)是常微分方程
\(\frac{\,dx}{\,dt}=V(x)\)关于初值条件
\(x(0)=x_0\)的解,
其中\(x_0\in R^n\)而且\(f\)使方程满足存在唯一性。
\(\{\phi(t,\cdot):R^n \rightarrow R^n:t\in R\}\) 构成一个动力系统。
特别是映射\(f(x):=\phi(1,x)\)导出一个离散动力系统。


在物理上,可以把它看成是物体在\(R^n\)空间中的运动,
而且在此空间中的每一点x处的速度是已经被规定好的且与时间无关。
\begin{proof}
1、\(R^n\)上面有拓扑结构,
从而\(R^n\) 是一个拓扑空间。

我们通过引入\(R^n\)中的标准拓扑来说明这一点。

2、\(\phi(t,x)\)是集合X上的一个流。

首先,我们来验证平移不变性,
即如果\(x=\phi(t,t_0,x_0)\)是方程的解,
那么对于与任意的常数\(\tau\),
\(x'=\phi(t+\tau,t_0,x_0)\)也是他的解。
\[\frac{\,dx'}{\,dt}
=\frac{\,d(\phi(t+\tau,t_0,x_0))}{\,dt}
=\frac{\,d(\phi(t+\tau,t_0,x_0))}{\,d(t+\tau)}\times \frac{\,d(t+\tau)}{\,dt}
=V(x)\times 1
=V(x)\]
这样,我们就验证了平移不变性。
我们来看两个两个解\(\phi(t-t_0,0,x_0)\) 和\(\phi(t,t_0,x_0)\),
他们满足相同的初值条件\(t_0.x_0\), 所以就有
\[\phi(t-t_0,0,x_0)=\phi(t,t_0,x_0)\]
再根据平移不变性,
\[\phi(t-t_0,0,x_0)=\phi(t,0,x_0)\]
也就是说\(\phi(t,0,x_0)\)代表了一切初始点在\(t_0.x_0\)的解的性质,
所以,也简记为\(\phi(t,x_0)\).

然后,我们取定一个初值\(x_0\),我们来验证它的群性质。
显然\(\phi(0,x_0)=x_0\).
考虑\(\phi(t,\phi(s,x_0))\)和\(\phi(t+s,x_0)\)
当t=0时,他们都取到了初值\(x_1\) 和\(x_2\),
\[x_1=\phi(0,\phi(s,x_0))=\phi(s,x_0)\]
\[x_2=\phi(0+s.x_0)=\phi(s,x_0)]
% 所以对于 $\forall s$ ,他们的初值是相同的,
根据解的存在唯一性,我们有
% $$ \phi(t,\phi(s,x_0))= \phi (t+s,x_0) $$

当\(x_0\) 取不同的点时,
可能对应着不同的积分曲线,
对应着不同的轨道,
得到不同的\(\phi(t,x_0)\).
我们让\(x_0\) 跑遍整个集合X,
就可以得到,
\[\phi(0,x)=x\]
\[\phi(t,\phi(s,x))=\phi(t+s,x)\]
到此,我们验证了\(\phi(t,x)\) 是一个流。

3、\(\phi(t,x)\) 连续
下面说明\(\phi(t,x)\) 在\((t_0,x_0)\) 点连续。
根据三角不等式,我们有

\begin{align*}
   & \left| \phi(t,x)-\phi(t_0,x_0)\right| \\
  = & \left| \phi(t,x)-\phi(t,x_0) +\phi(t,x_0) -\phi(t_0,x_0) \right|  \\
  \leq &\left| \phi(t,x)-\phi(t,x_0)\right| +\left|\phi(t,x_0) -\phi(t_0,x_0) \right|
\end{align*}

对于任意给定的\(\epsilon > 0\),
由解\(\phi(t,x_0)\)在\(t_0\)的连续性,
存在\(\delta_1 (< \delta)\),
使当\(|t-t_0|<\delta_1\)时,
\[\left|\phi(t,x_0) -\phi(t_0,x_0) \right|< \frac{\epsilon}{2}\]

根据解对初值的连续依赖性,
\[\left| \phi(t,x)-\phi(t,x_0)\right| \leq |x-x_0|e^{Lt}\]
其中L是Lipschitz常数。
注意到,\(|t-t_0|< \delta_1\),\(|t|-|t_0| \leq |t-t_0| < \delta_1\),\(|t| \leq |t_0| + \delta_1\).
此时,\(|x-x_0|e^{Lt} \leq |x-x_0|e^{L(|t_0|+\delta_1)}\),
其中\(e^{L(|t_0|+\delta_1)\)是一个常数。
于是,根据范数的连续性,存在\(\delta_2\),
使当\(|x-x_0|< \delta_2,x\in X\) 时,
\(|x-x_0|e^{L(|t_0|+\delta_1)}< \frac{\epsilon}{2}\)
即,
\[\left| \phi(t,x)-\phi(t,x_0)\right|< c\]

故,\(\left| \phi(t,x)-\phi(t_0,x_0)\right|< \frac{\epsilon}{2} +\frac{\epsilon}{2} =\epsilon\)
\end{proof}


\subsubsection {解对初值的连续依赖性的证明}
\begin{thm}[Gronwall不等式]
设\(x(t),f(t)\)为区间\([t_0,t_1]\) 上的实连续函数,
\(f(t) \beq 0\),
若存在实常数g,使得
\[x(t) \leq	g + \int_{t_0}^{t}f(\tau)x(\tau)\,d\tau,t\in [t_0,t_1],\]
则
\[x(t) \leq g exp(\int_{t_0}^{t}f(\tau)\,d\tau),t \in [t_0,t_1],.\]
\end{thm}

\begin{thm}[推广的Gronwall不等式]
设\(x(t),g(t)\)为区间\([t_0,t_1]\) 上的实连续函数,
函数\(f(t) \beq 0\)在区间\([t_0,t_1]\)上可积,
他们满足,
\[x(t) \leq g(t) + \int_{t_0}^{t}f(\tau)x(\tau)\,d\tau,t \in [t_0,t_1],\]
则当\(t \in [t_0,t_1]\)时,
\[x(t) \leq g(t) + \int_{t_0}^{t}f(\tau)g(\tau)exp(\int_{\tau}^{t}f(s)\,ds)\,d\tau\]
\end{thm}

\begin{thm}
设方程有解\(x(t,y_0)\)与\(x(t,z_0)\),
他们都在区间\([t_0,t_1]\)上存在,
则对一切\(t\in [t_0,t_1]\)有,
\[x(t,y_0)-x(t,z_0 \leq |y_0-z_0|e^{L(t-t_0)},\]
其中L是常数。
\end{thm}

\begin{thm}
设方程有解\(x(t,y_0)\),
在区间\([t_0,t_1]\)上存在,
则存在\(y_0\)的领域U,
使得当\(z_0 \in U\)时,
就有方程的唯一一个解\(x(t,z_0)\)也在区间\([t_0,t_1]\)上有定义,
则对一切\(t\in [t_0,t_1]\)有,
\[x(t,y_0)-x(t,z_0 \leq |y_0-z_0|e^{L(t-t_0)},\]
其中L是常数。
\end{thm}

\begin{center}
  \textbf{证明顺序}
\end{center}


\begin{center}
\begin{tikzpicture}
  [node distance=1cm,
  start chain=going below,]
     \node[punktchain, join]  {Gronwall's inequality};
     \node[punktchain, join] {Generalized Gronwall's inequality};
     \node[punktchain, join]  {Theorem3};
     \node[punktchain, join]   {Theorem4};
  \end{tikzpicture}
\end{center}

\subsection{符号动力系统}
设\(\Sigma\)是由一切双边序列,
\(S=(\dots,s_{-2},s_{-1},s_{0},s_1,s_2,\dots)\)组成的集合,
\(s_0\)表示零位坐标元素,
映射,\(\delta:(\dots,s_{-1},s_0,s_1,\dots)\Rightarrow (\dots,d_{-1},s_0,s_1,\dots)\)
易见(\dots,a_{-1},a_0,a_1,\dots)是周期点.

1.\(\Sigma(N)\)是拓扑空间
通过引入距离,
\(d(x,y)=\Sigma_limits_{n\t0\infty}^{\infty}\frac{d(x_n,y_n)}{2^{|n|}}\)
其中,
\(d(x_n,y_n)=\)
2.\(\delta\)是同胚,

3.柱集

4.可数拓扑基


\subsection{微分动力系统}

流形的概念是欧式空间的推广,
粗略的说,
流形在每一点的近傍和欧式空间的一个开集是同胚的,
因此在每一点的近傍可以引入局部坐标系,
流形是一块块"欧式空间"粘起来的结果。

\begin{Defination}
设M是Housdorff空间,
若对任意一点\(x\in M\),
都有x在M中的一个领域U同胚于欧式空间\(R^m\)的一个开集,
则称M是一个m维\textbf{流形}。
\end{Defination}

设同胚映射为\(\phi_{U}:U\leftarrow \phi_{U}\),
则称\((U,\phi_{U})\)是M的一个坐标卡。

因为\(\phi_{U}\)是同胚,
对任意一点\(y \in U\),
可以把\(\phi_{U}(y)\in R^{m}\)的坐标定义为y的坐标,
即命
\[u^i=(\phi_U(y))^i,y\in U,i=1,2,\dots,m.\]
我们称\(u^1(1\leq i \leq)\)为点\(y \in U\)的局部坐标。

设\((U,\phi_U)\)和\((V,\phi_V)\)是M的两个\textbf{坐标卡},
若\(U\cap V \neq \emptyset \),
则\(\phi_U(U\cap V)\)和\(\phi_V(U\cap V)\)都是\(R^m\)中的两个开集。
\[\phi_V \comp \phi_U^{-1}:\phi_U(U\cap V) \leftarrow \phi_V(U\cap V)\]
建立了两个欧式空间上开集的同胚,
其逆映射为\(\phi_U \comp \phi_V^{-1}\)。

因为它们是从欧式空间的一个开集到另一个开集的映射,
所以用坐标表示时,
\(\phi_V \comp \phi_U^{-1}\)和\(\phi_U \comp \phi_V^{-1}\)分别表示欧式空间的开集上的m个实函数。

\[\phi_V \comp \phi_U^{-1}:y^i=f^i(x_1,x_2,\dots,x^m),x \in \phi_U(U\cap V \]
\[\phi_U \comp \phi_V^{-1}:x^i=g^i(y_1,y_2,\dots,y^m),y \in \phi_V(U\cap V)\]

设\(f\)是实数集\(U\in R^m)上的实函数,
如果\(f\)的直到\(k\)次的偏导数存在且连续,
则称\(f\)是r次可微的,或者称\(f\) 是\(C^r\)的。

如果\(y^i=f^i(x_1,x_2,\dots,x^m)\) 和\(x^i=g^i(y_1,y_2,\dots,y^m)\)都是\(C^r\)的。
(\(U \cap V \neq \emptyset \)),或者\(U \cap V \neq \emptyset \),
我们称(\(U,\phi_U)\)和\((V,\phi_V)\)是\textbf{\(C^r\)相容的}。

\begin{Defination}
设M是一个m维流形,如果在M上给定了一个坐标卡集
\( A= \{ (\(U,\phi_U)\), (\((V,\phi_V)\), (\(W,\phi_W)\),\dots \} \),
满足下列条件,则称A是M的一个\textbf{微分结构}:
\begin{enumerate}
                        \item [\((\romannumeral 1)\)]  \(\{U,V,W,\dots\}\)是M的一个开覆盖;
                    \item [\((\romannumeral 2)\)]  属于A的任意两个坐标卡都是\( C^r-\)相容的;
                        \item [\((\romannumeral 3)\)]  A是极大的,
                        即对于M的任意一个坐标卡\((\bar{U}:\phi_{\bar{U}})\),
                        若与属于A的每一个坐标卡都是\(C^r\) 相容的,
                        则它自身属于A。
                \end{enumerate}

\end{Defination}

\begin{Defination}
若在M上给定了一\(C^r\)微分结构,则称M是一个\(C^r\)\textbf{微分流形}。
\end{Defination}

\begin{Defination}
如果X是一个\(C^r\)微分流形,
\(\phi\)是\(C^r\)映射,
则称\(\phi\)诱导出的动力系统是\textbf{微分动力系统}。
\end{Defination}


\section{轨道与周期点}

\subsection{轨道}
称集合
\[Orb_f(x)=\{f^k(x):k\in Z\}\]
为f过\(x\in X\)的\textbf{轨道}。

\subsection{周期点}
如果存在自然数p,
使得\(f^p(x)=x\),则称x为f的\textbf{周期点}。

\begin{thm}
周期轨道都是有限轨道,有限轨道都是周期轨道。
\end{thm}

\begin{proof}
设\(Orb_f(x)=\{f^k(x):k\in Z\}\)是有限轨道,
故\(Orb_f(x)\)只有有限个互不相同的点。
则存在一个自然数N,
\(O=\{f^0(x),f^1(x),\dots,f^N(x)\}\) 包含了所有这些点。
则\(f^{N+1}(x)\)必与O中的某一个或几个相等。
不妨设\(f^{N+!}(x)=f^k(x),0 \leq k \leq N\),
此时,\(f^{N+!-k}(x)=x\),即x为周期点。

设\(Orb_f(x)=\{f^k(x):k\in Z\}\)是周期轨道,
不妨设周期为p,即
\[f^p(x)=x,f^l(x)\neq x ,\forall l=1,2,\dots,p-1\]
此时,\(\forall k \in Z\),记\( k mod(p)=b,0 \leq b \leq p\),则
\[f^k(x)=f^b(x)\in \{f(x),f^2(x),f^3(x),\dots,f^p(x)\},\]\
故\(Orb_f(x)\)是有限轨道。

\end{proof}


\section{极限点与非游荡点}
\subsection{极限点}
一个点\(x \in X\)的正半轨
\[x,f(x),f^2(x),\dots\]
一般都是不收敛的(若收敛,其极限必为不动点),
但总有许多子序列收敛。

\begin{Defination}
称\(y\in X\)为X的一个\textbf{\(\omega\)极限点},
如果存在正整数的一个子序列,\(n_i rightarrow \infty\)使得
\[f^{n_i}(x) \rightarrow \infty y。\]
称X的全体\(\omega\)极限点为X的\(\omega-\)极限集。
\end{Defination}



\begin{Defination}
称\(y \in X\)为X的一个 \textbf{\(\alpha\)极限点},
如果有正整数的一个子序列
,
\(n_i \rightarrow \infty\),
使得
\[f^{-n_i} \rightarrow y,\]
称x的全体\(\alpha\)极限点为\(\alpha-\)极限集。
\end{Defination}

\begin{figure}
  \centering
  % Requires \usepackage{graphicx}
  \includegraphics[width=8cm]{极限集.png}\\
  \caption{极限点}\label{}
\end{figure}

hopf分岔是一种常见的分岔,或产生极限环,在极限环里面的点,是向外发散,极限环外面的点,向内收敛,但都是向极限环无限的逼近。


\begin{figure}
  \centering
  % Requires \usepackage{graphicx}
  \includegraphics[width=8cm]{113.png}\\
  \caption{hopf分岔}\label{}
\end{figure}

前面的概念都是在度量空间中考虑的,
现在怎么把极限的概念推广到一般的欧式空间,
我们引入如下的概念。

\begin{Defination}
集合
\[\omega(f)=\cap\limits_{n\in N}\bar{\{f^k(x):k \beq n\}}\]
\[\alpha(f)=\cap\limits_{n\in N}\bar{\{f^{-k}:k \beq n\}}\]
分别称为轨道\(Orb_f(x)\)的\textbf{\(\omega\)极限集},
和\textbf{\(\alpha\)极限集}。
\[L(f)=\cap_{x\in X}\omega_f(x)\cup \alpha_f(x)\]
称为f的极限集。
\end{Defination}

\subsection{非游荡点}
我们知道,如果\(f^p(x)=x\),我们把x 叫做周期点。
有时候,
这个条件对于我们来说太过苛刻,
这时候,找一个比它弱一点的条件,
就是\(f^k(x)\)还在X的附近,
没有走远,
我们把这个概念抽象出来,就得到下面的概念。

\begin{Defination}
点\(x\in X\)称为X的游荡点,
如果存在x的领域U,使得,
\[f^k(U)\cap U =\emptyset ,\forall k \in Z\backslash\{0\}。\]
不是游荡点的点称为\textbf{非游荡点}。
非游荡点的集合记为\(\Omega(f)\).
\end{Defination}

\begin{figure}
  \centering
  % Requires \usepackage{graphicx}
  \includegraphics[width=8cm]{非游荡点集.png}\\
  \caption{非游荡点集}\label{}
\end{figure}

\begin{thm}
        \begin{enumerate}
                        \item [\((\romannumeral 1)\)]  \(\omega_f(x),\alpha_f(x),\Omega(f)\) 都是闭的;
                    \item [\((\romannumeral 2)\)]  \(Per(f)\subset L(f) \subset \Omega(f)\)
                        \item [\((\romannumeral 3)\)]  当X 是紧空间时,\(\omega_f(x),\alpha_f(x)\),和\(\Omega(f)\)都非空。
        \end{enumerate}
\end{thm}

\begin{proof}
1.任意多个闭集的交是闭的。
故对于\(\omega_f(x)\)和\(\alpha_f(x)\),
因为闭包是闭,
由定义知,
\(\omega_f(x)\)和\(\alpha_f(x)\)都是任意多个闭集的交,
故是闭集。
任意多个开集的并是开集,
要证所有非游荡点总是闭集,
即证所有游荡点的集合是开集。
设x为游荡点,
由定义知,
存在x的领域U,使
\[f^k(U)\cap U =\emptyset ,\forall k \in Z\backslash\{0\}。\]
则对于U中任意一点也都是游荡点。
领域是开集,故所有的游荡点构成的集合为开集。
综上知,\Omega(f)是闭集。

2、Per(f)是所有周期点的集合。
\(\forall x \in Per(f)\),
则x为f的周期点,
不妨设x为p-周期点,
p为正整数,
则\(f^p(x)=x\).
\(\forall n \in N\),当\(k beq 0\) 时,总可以使\(kp \beq n\)(k充分大),使得\(f^{kp}(x)=x\in\omega_f(x)\).
当\(k < 0\)时,总可以使 \(-kp \beq n\)(k充分小),使得\(f^{-kp}(x)=x\in\alpha_f(x)\)
而\(L(f)=\cap_{x\in X}\omega_f(x)\cup \alpha_f(x)\), 故\(x\in L(f)\),从而
\(Per(f)\subset L(f)\).
下面证明\(L(f) \subset \Omega(f)\)
\(\forall x_0 \in L(f)\),
则\(x_0\in \omega_f(x)\)或\(x_0\in \alpha_f(x)\).
当\(x_0\in \omega_f(x)\)时,即\(x_0\)为\omega - 极限点,下证\(x_0\)为非游荡点。
根据定义,存在正整数的一个子序列,\(n_i rightarrow \infty\)使得
\[f^{n_i}(x) \rightarrow \infty x_o,x\in Z\]
即\(\forall \epsilon >0,\exists I>0,\forall i>I,
\left|f^{n_i}-x_0 \right|<\epsilon        \).
即当\(i>I\)时,\(f^{n_i}(x) \in \delta(x_0,\epsion)\)故对\(x_0\)的任意领域U,都存在不同的正整数k,l,
使得,\(f^k(x),f^l(x)\in U\),不妨设\(k>l\),即
\[x \in f^{-k}(U),x \in f^{-l}(U)\]
故\(f^{-k+l}(U)\cap U \supset f^{l}(f^{-k} \cap f^{-l}(U))\neq \emptyset              \)
(\(x\in f^{-k}(U)\)且 \(x\in f^{-l}(U)).
其中,\(-k+l<0\).

当\(x_0\in \alpha_f(x)\)时,即\(x_0\)为 \alpha - 极限点,根据定义
有正整数的一个子序列,
\(n_i \rightarrow \infty\),
使得
\[f^{-n_i} \rightarrow y,\]
即对x_0的任意领域U,都存在不同的正整数k,l,
使得,使得,\(f^{-k}(x),f^{-l}(x)\in U\),不妨设\(k>l\),即
\[x \in f^{k}(U),x \in f^{l}(U)\]
故\(f^{k-l}(U)\cap U \supset f^{-l}(f^{k} \cap f^{l}(U))\neq \emptyset              \)
(\(x\in f^{k}(U)\)且 \(x\in f^{l}(U)).
其中,\(-k+l<0\).
综上知,\(f^m(U)\cap U \neq \emptyset,\forall m\in z\backslash{0}\),
即x_0是非游荡点,从而\(L(f) \subset \Omega(f)\).

3、X为紧的度量空间,\(\omega_f(x)\subset X,\alpha_f(x) \subset X \),\(\Omega(f) \subset X\)
列紧集\(M\subset X\)中每个点列都有收敛子序列收敛于一点\(x\in M\).
又因为,列紧空间中任意子集都是列紧集。
故\(\omega_f(x),\alpha_f(x)\),和\(\Omega(f)\)都是列紧集。
又列紧集的定义,总有一个点是在集合当中,故集合非空。
从而\(\omega_f(x),\alpha_f(x)\), 和\(\Omega(f)\) 都非空。
\end{proof}

\begin{thm}
\(Orb_f(x),Fix(f),Per(f),\omega_f(x),\alpha_f(x),L(f),\Omega(f)\) 都是f的不变集。
\end{thm}

\begin{proof}

1、\(Orb_f(x)=\{f^k(x):k\in Z\}\)

\(\forall y \in Orb_f(x)\),即\(\forall k \in Z,\exists y=f^{k}(x)\in Orb_f(x)\).
\(z=f^l(y)=f^l(f^k(x))=f^{l+k}(x),l+k\in Z\),
即\(z\in Orb_f(x)\),
故\(Orb_f(x)\)为不变集。

2、Fix(f)为f在X上所以不动点的集合。
\(\forall x \in Fix(f),f(x)=x,即f(Fix(f))=Fix(f)\),
故Fix(f)为不变集。

3、Per(f)为f在X上所有周期点的集合。
\(\forall x\in Per(f)\),即存在\(p\in N\),
使得\(x=f^p(x)\).
\(Orb_f(x)={f^k(x):k\in Z}\)
\(\forall y \in Orb_f(x)\),
即\(\forall k \in Z,y=f^k(x)\in Orb_f(x)\),
下证\(y\in Per(f)\).
f^p(y)=f^p(f^k(x))=f^{p+k}(x)=f^k(f^p(x))=f^k(x)=y
故\(y\in Per(f)\),
即Per(f)为不变集。

4、\(\omega(f)=\cap\limits_{n\in N}\bar{\{f^k(x):k \beq n\}}\)
\(\forall y\in \omega_f(x)\),即存在\(N>0\)(充分大),当\(n>N\)时,
\(y=f^k(x),k\beq n>N\)(充分大)。
Orb_f(y)={f^l(y):l\in Z}
\(\forall Z \in Orb_f(y)\),
即\(\forall l\in Z,\)
有\(z=f^l(y)\in Orb_f(y)\),下证\(z\in \omega_f(x)\)
\(z=f^l(y)=f^l(f^k(x))=f^{l+k}(x),l+k \beq n+l >N\)
(n可充分大于N)
故\(z\in \omega_f(x)\),
从而\(\omega_f(x)\)为不变集。

5、\(\alpha(f)=\cap\limits_{n\in N}\bar{\{f^{-k}:k \beq n\}}\)
\(\forall y\in \alpha_f(x)\),即存在\(N>0\)(充分大),当\(n>N\)时,
\(y=f^{-k}(x),k\beq n>N\)(充分大)。
Orb_f(y)={f^l(y):l\in Z}
\(\forall Z \in Orb_f(y)\),
即\(\forall l\in Z,\)
有\(z=f^l(y)\in Orb_f(y)\),下证\(z\in \alpha_f(x)\)
\(z=f^l(y)=f^l(f^{-k}(x))=f^{l-k}(x),l-k \beq n-l >N\)
(n可充分大于N)
故\(z\in \alpha_f(x)\),
从而\(\alpha_f(x)\)为不变集。

6、首先我们来证明\(f(\Omega(f))\subset \Omega(f)\).\\
任取\(x\in \Omega(f)\),
考虑f(x)的任意一个领域U,
则\(V:f^{-1}(U)\)是x的领域,
从而有\(f^n(V) \cap V \neq \emptyset\)
从而,
\[f^n(U)\cap U=f^{n+1}(U)\cap f(U)\supset f(f^(U)\cap U)\neq \emptyset\]
从而,\(f(x)\in \Omega(f)\)
从而,\(f(\Omega(f))\subset \Omega(f)\).

然后我们来证明\(\Omega{f} \subset f(\Omega{f})\).
下证\(\Omega(f^{-1})=\Omega(f)\).
\(\forall x\in \Omega (f),\forall x \in U,f^{k}(U)\cap U \neq \emptyset\)
\(f^{-k}(f^k(U)\cap U)= U\cap f^{-k}(U)\),
故\(x\in \Omega(f^{-1})\).
\(\forall x\in \Omega (f^{-1}),\forall x \in U,f^{-k}(U)\cap U \neq \emptyset\)
\(\emptyset=f^{k}(f^{-k}(U)\cap U)= U\cap f^{-k}(U)\),
故\(x\in \Omega(f)\)
故\(\Omega(f^{-1})=\Omega(f)\).



将前半部分的结论带到\(f^{-1}\),就有,
\(f^{-1}(\Omega(f^{-1})\subset \Omega(f^{-1})\),
故,\(f^{-1}(\Omega(f)\subset \Omega(f)\),
故\Omega(f)\subset f(\Omega(f))

综上知,\(f(\Omega(f))\=\Omega(f)\),即\Omega(f) 为不变集。

\end{proof}
