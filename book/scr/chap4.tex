\chapter{双曲结构}
设$f$为 $C^r(r>1)$Banach流形$X$上的$C^r$动力系统,
$f(x_0)=x_0$.
\begin{defination}
  不懂点$x_0$称为是双曲的,
  如果其切映射$A=Df(x_0):T_{x_0}X \to T_{x_0}X$是可逆映射且具有不变的直和分解,既
  \begin{equation}
    \label{eq:0202}
T_{x_0}X= E^s \oplus E^u, AE^s = E^s ,AE^u =E^u,
\end{equation}
且满足扩张性和收缩性,既对 $\forall k \in Z_{+}$,
\begin{equation}
  \label{eq:0203}
  \begin{aligned}
    || A^k x|| \leqslant c \lambda^k ||x||, \forall x \in  E^s,\\
    ||A^{-k}x|| \leqslant c \lambda^k ||x||, \forall x \in E^u,
  \end{aligned}
\end{equation}
其中$c>0,0<\lambda<1$为常数.
\end{defination}
\section{Hartman线性化}
\subsection{离散动力系统的Hartman线性化定理}
\begin{defination}
  拓扑空间$X$上同胚$f$和拓扑空间$Y$上同胚$g$称为是拓扑共轭的,
  简记为$f \sim g$,
  如果存在同胚 $h : X \to Y$,
  使得
 $ h  \comp g = g \comp f$
\end{defination}


\begin{defination}{Hartman-Crobman线性化定理}
  设$f$为$C^r(r \geqslant 1 )$Banach流形$X$上的$C^r$动力系统,
  $x_0 \in X$ 为 $f$ 的双曲不动点.
  则存在$x_0$的领域 $U \subset X $和 $O$的领域 $V  \subset T_{x_0}X$,
  使得$f|_{U}$与$Df(x_0)|_v$拓扑共轭.

\end{defination}
\subsection{微分方程的线性化定理}
\begin{ode}
  \label{equ:4.1}
&  \dxdt = ax+by,\\
&  \dydt = cx+dy,
\end{ode}
其中,a,b,c,d是实数,
\begin{ode}
  \label{equ:4.2}
  & \dxdt =ax+by+ \Phi(x,y),\\
  & \dydt =cx+dy+ \Psi(x,y),
\end{ode}
其中,a,b,c,d是实数,$\Phi(0,0)=\Psi(0,0)=0$,
且在原点的领域内$\Phi(x,y),\Psi(x,y)$对$x,y$连续,
还满足解的唯一性条件.
\begin{theorem}
  设$O(0,0)$是方程组\ref{equ:4.1}的稳定(不稳定)吸引子,
  且方程组
\end{theorem}

\begin{theorem}
  设$O(0,0)$是方程组\ref{equ:4.1}的稳定(不稳定)焦点,
  另外条件1成立,
  则$O(0,0)$也是方程组的稳定(不稳定)焦点.
\end{theorem}

\begin{theorem}
  设O(0,0)是方程组\ref{equ:4.1}的正常结点,
  设特征根$|\lambda_2|>|\lambda_1|$.

  \begin{description}
  \item [1] 如果条件1满足,则方程组在奇点附近的轨线都沿特殊方向$\theta =0,\frac{\pi}{2},\pi ,\frac{3\pi}{2}$进入奇点O.
  \item [2] 如果条件2也满足,则沿$\theta=\frac{\pi}{2},\frac{3\pi}{2}$各只有一条轨线进入奇点O.
  \end{description}
\end{theorem}

\begin{theorem}
  对系统有:
  \begin{enumerate}
  \item[enu1] 当线性方程组\ref{equ:4.1}的奇点O是焦点时,
    如果方程组\ref{equ:4.2}的附加项$\Phi,\Psi$满足条件1,
    则奇点O仍是\ref{equ:4.2}的焦点,
    且稳定性不变;
  \item[enu2] 当奇点O是\ref{equ:4.1}的鞍点或正常结点时,
    如果$\Phi,\Psi$满足条件1和2,
    则相应的奇点O仍分别是\ref{equ:4.2}的鞍点和正常结点.
    且对正常结点来说,不改变稳定性;
  \item[enu3]当奇点O是\ref{equ:4.1}的退化结点时,
    如果$\Phi,\Psi$满足条件1和2,
    则奇点O仍是\ref{equ:4.2}的i临界结点,
    且不改变稳定性.
  \item[enu3]
  \end{enumerate}
\end{theorem}

\begin{theorem}{S.Sternberg,线性化定理}
  设\ref{equ:4.2}方程组中的$\Phi,\Psi \in C^\infty$,
  而且
  \begin{equation}
    \label{eq:0204}
\Phi,\Psi =O(r^2),r\to0,
  \end{equation}
  方程组~(\ref{equ:4.1})的特征根$\lambda_1$和$\lambda_2$满足:
  \begin{equation}
    \label{eq:0205}
    \lambda_k \neq m_1\lambda_1 +m_2\lambda_2,k=1,2,
  \end{equation}
  其中$m_1,m_2$为非负整数,
  且$2\leq m_1+m_2,$
  则存在$c^\infty$微分同胚
  可使得方程组~(\ref{equ:4.2})线性化.
\end{theorem}

\section{双曲线性映射}

\begin{theorem}
  设(X,||.||)为Bananch空间.
  可逆线性映射A \in L(X,X)是双曲线性映射的充分必要条件为
  它的存谱集 $\sigma(A)$与复平面上的单位圆$S^1$不相交,
  既$\sigma(A) \cap S^1 = \emptyset$.
\end{theorem}
\section{稳定流形与不稳定流形}