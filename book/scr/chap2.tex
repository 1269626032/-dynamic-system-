\part{结构稳定性}
\section{离散动力系统的结构稳定性}

\subsection{全局结构稳定性}

\begin{definition}
    \(C^{r}\) 流形X上的动力系统f称为是结构稳定的,
    如果存在f在\(C^{r}\) 拓扑中的领域U,
    使得,对任意\(g\in U\)
    都与f拓扑共轭.
\end{definition}

显然,\(C^{r}\) 结构稳定则是\(C^{r+1}\) 结构稳定,
\(C_1\) 结构稳定性通常直接称为结构稳定性.

\begin{theorem}
    \(C^{r}\) 结构稳定则是\(C^{r+1}\) 结构稳定.
\end{theorem}
\begin{proof}
\end{proof}


\subsubsection{f的\(C^{r}\)拓扑}

固定M的允许坐标领域的一个有限覆盖\((U_i,\phi_i),i=1,2,\dots,N\)
称一列\(C^{r}\)微分同胚\(f_n\)在\(C^{r}\)意义下,
收敛到一个\(C^{r}\)微分同胚\(f\),
如果,对所有\(1\leq i,j \leq N\),
局部坐标表示,
\(\phi_j f_n \phi _i^{-1}\)连同直到r阶偏导数,
在使这些局部表示有意义的电上,
一致收敛到
\(\phi_jf\phi_i^{-1}\)
及其所对应的偏导数.

\subsubsection{f的范数的定义}


\subsection{局部结构稳定性}

\begin{definition}
    设\(U\subset X\)是开集,
    且\(f\in C^{r}(U,X)\)是
    到其相集的同胚,
    f在\(p \in U\)处被称为是\(C^{r}\)局部结构稳定的,
    如果存在p点的领域\(V\subset U\)及
    f在\(C^{r}(U X)\)中的领域\(W\),
    使得,对任意\(g\in W\),
    都与某点\(q\in V\)
    处,与\(f\)在\(p\)点处局部拓扑共轭.
\end{definition}


\begin{definition}
    f在点\(p\in X\)和g在点\(q\in Y\)
    称为局部拓扑共轭的,
    如果存在p,q的开领域
    \(U\subset X\),\(V \subset Y\),
    和同胚
    \(h:\)
    使得

\end{definition}

\begin{theorem}
    设X是Bananch空间,
    \(U\subset X\)是O的开领域,
    若O是\(f\in C^{1}(U,X)\)的双曲不动点,
    则f在O附近是局部结构稳定的.
\end{theorem}

\begin{proof}
    第一步,首先证明,存在O的领域\(W\subset U\)
    和f的\(C^1\)领域\(V\)使得,
    使得,任意\(g \in V\)在W中有唯一的不动点c,
    并且c是g的双曲不动点.
    事实上,X上双曲线性映射全体\(H(X)\)构成可逆有界线性映射空间
    \(L(X)\)上的开集,
    故存在\(\delta\)使\(A=Df(O)\)的\(\delta -\)领域属于\(H(X)\).
    由Df的连续性,
    存在\(\alpha >0\) 使得当
    \(||x||<\alpha \)时,

\end{proof}

\section{连续动力系统的结构稳定性}


\subsection{全局结构稳定性}

%\begin{definition}
 %   所谓X中某一系统
  %  \[\frac{dx}{dt}=P(x,y),\frac{dy}{dt}=Q(x,y)\]
%的\(\epsilon-\)临近系统,
%是指满足条件
%\[X-P+\left| \frac{\}{}\]

\end{definition}


\begin{theorem}[安德罗洛夫-旁特里雅金定理]
    在\(X(\Omega)\)中系统结构稳定的重要条件是:
    (1)它只有有限个奇点,
    而且所有的奇点都是双曲的.
    (2)它只有有限条闭轨,
    而且所有的闭轨都是双曲的.
    (3)它没有从鞍点到鞍点的轨线.
\end{theorem}

\begin{theorem}[皮邦图定理]
  设\(M^2\)是二维可定向的紧流形,
  \(f\in C^1(M^2)\),
  则向量场为结构稳定的充要条件是:
  (1)系统有有限个平衡点和闭轨,
  且他们都是双曲的;
  (2)系统不存在从鞍点到鞍点的轨线;
  (3)系统的非游荡点集仅由平衡点和闭轨组成.

\end{theorem}

\begin{theorem}[皮邦图稠密性定理]
  设\(M^2\)是二维可定向的紧流形,
  记\(C^1(M^2)\)中一切结构稳定的向量场的子集为
  \(\Sigma\),
  则\(\Sigma\)在\(C^1(M^2)\)中是开的且是稠密的.
\end{theorem}

\subsection{局部结构稳定性}
